
\documentclass[12pt]{beamer}


\usepackage[english]{babel}
\usepackage{amsmath}               
\usepackage{amsfonts}              
\usepackage{amsthm}                
\usepackage{amssymb}
\usepackage{graphicx}
\usepackage{color}
\usepackage{../mysymbols}
\usepackage{tikz}
\usefonttheme{serif}
\usetikzlibrary{positioning}
\usetikzlibrary{calc}
\usetikzlibrary{decorations.pathreplacing}
\usetikzlibrary{matrix}
\usetikzlibrary{intersections}
\usetikzlibrary{backgrounds}
\usetikzlibrary{fit}
\usetikzlibrary{arrows}
\usepackage{algorithmic}
\usepackage{algorithm}



\usetheme{Boadilla}
\usecolortheme{lily}
\setbeamertemplate{navigation symbols}{}
\setbeamertemplate{footline}{}
\setbeamercovered{transparent}
\setbeamertemplate{headline}
{
  \leavevmode%
  \hbox{%
  \begin{beamercolorbox}[wd=.4\paperwidth,ht=2.25ex,dp=1ex,center]{author in head/foot}%
    \usebeamerfont{author in head/foot}\insertsection
  \end{beamercolorbox}%
  \begin{beamercolorbox}[wd=.6\paperwidth,ht=2.25ex,dp=1ex,right]{title in head/foot}%
    \url{http://defeo.lu/towers}\hfill
    \insertframenumber{} / \inserttotalframenumber\hspace*{1ex}\hspace*{2mm}
  \end{beamercolorbox}}%
  \vskip0pt%
}
\setbeamertemplate{frametitle}{
	\begin{centering}
		\textbf{\small \insertframetitle}
		\par
	\end{centering}
}


\newcommand{\Cyc}{\Phi}  % Cyclotomic polynomial
\newcommand{\bb}{\mathbf{B}}  
\newcommand{\uu}{\mathbf{U}}  % bases for lift and push
\DeclareMathOperator{\Res}{Res}


\newtheorem{proposition}[theorem]{Proposition}

\renewcommand{\algorithmicrequire}{\textbf{Input:}}
\renewcommand{\algorithmicensure}{\textbf{Output:}}




\title
{\bf Fast algorithms for {\Large$\mathbf{\ell}$}-adic\\ towers over finite fields}


\author{
	Luca De Feo\inst{1},\and\hspace*{-3mm} 
	Javad Doliskani\inst{2},\and\hspace*{-3mm} 
	\'Eric Schost\inst{2}
}

\institute{
  \inst{1}Laboratoire PRiSM, Universit\'e de Versailles \and
  \vspace*{-2mm}
  \inst{2}Department of Computer Science, Western University
}

\subject{Computer Algebra}

\date{ISSAC, June 28, 2013}

\pgfdeclareimage[height = 5mm]{westernlogo}{Western_Logo_CMYK}
\pgfdeclareimage[height = 6mm]{versailleslogo}{uvsq-logo}

\begin{document}

\setbeamertemplate{itemize subitem}[triangle]
\setbeamercolor{itemize subitem}{fg=red}

\begin{frame}
	\thispagestyle{empty}
	\titlepage

	\begin{flushright}
		\pgfuseimage{westernlogo}\hspace*{1mm}\pgfuseimage{versailleslogo}
	\end{flushright}

        \begin{center}
          \scriptsize
          Find these slides, the paper and the source code on \url{http://defeo.lu/towers}
        \end{center}
\end{frame}

%**********************************************************
%**********************************************************

\section{Introduction}

\begin{frame}
	\frametitle{Problem statement}
	
	\tikzset{
		dotstyle/.style={circle, inner sep = 1.2pt, outer sep = 4pt, fill = gray},
		edgetower/.style={thick},
		edgecomp/.style={thick, lightgray}
	}
	\begin{tikzpicture} [scale = 0.8] %, every node/.style={inner sep = 2pt, scale=0.8}]
		\coordinate (T2) at (-2, 0.5);
		\node (Fq) at (0, 0) {$\F_q$};
		\node (Fq2) at ($(Fq) + (T2)$) {$\F_{q^2}$};
		\node (Fq4) at ($(Fq2) + (T2)$) {$\F_{q^4}$};
		\node (Fq2l) at ($(Fq4) + (T2)$) {$\F_q^{(2)}$};
		%---------------------
		\coordinate (T3) at (-0.7, 2);
		\node (Fq3) at ($(Fq) + (T3)$) {$\F_{q^3}$};
		\node (Fq9) at ($(Fq3) + (T3)$) {$\F_{q^9}$};
		\node (Fq3l) at ($(Fq9) + (T3)$) {$\F_q^{(3)}$};
		%---------------------
		\coordinate (T5) at (0.7, 2);
		\node (Fq5) at ($(Fq) + (T5)$) {$\F_{q^5}$};
		\node (Fq25) at ($(Fq5) + (T5)$) {$\F_{q^{25}}$};
		\node (Fq5l) at ($(Fq25) + (T5)$) {$\F_q^{(5)}$};
		%---------------------
		\coordinate (Tl) at (2, 1);
		\node[blue] (Fql) at ($(Fq) + (Tl)$) {$\F_{q^\ell}$};
		\node[blue] (Fql2) at ($(Fql) + (Tl)$) {$\F_{q^{\ell^2}}$};
		\node[blue] (Fqll) at ($(Fql2) + (Tl)$) {$\F_q^{(\ell)}$};
		%---------------------
		\node[dotstyle] (dot1) at ($(Fq2) + (Fq3) - (Fq)$) {};
		\node[dotstyle] (dot2) at ($(Fq4) + (dot1) - (Fq2)$) {};
		\node[dotstyle] (dot3) at ($(Fq2) + (Fq5) - (Fq)$) {};
		\node[dotstyle] (dot4) at ($(Fq3) + (Fq5) - (Fq)$) {};
		\node[dotstyle] (dot5) at ($(Fq3) + (Fql) - (Fq)$) {};
		\node[dotstyle] (dot6) at ($(Fq5) + (Fql) - (Fq)$) {};
		%---------------------
		\draw 
			(Fq) 
				edge[edgetower] (Fq2)
				edge[edgetower] (Fq3)
				edge[edgetower] (Fq5)
				edge[edgetower, blue] (Fql)
			(Fq2)
				edge[edgetower] (Fq4)
				edge[edgecomp] (dot1)
			(Fq4)
				edge[edgetower, dotted] (Fq2l)
				edge[edgecomp] (dot2)
			(dot1)
				edge[edgecomp] (dot2)
			(Fq3)
				edge[edgetower] (Fq9)
				edge[edgecomp] (dot1)
				edge[edgecomp] (dot4)
			(Fq9)
				edge[edgetower, dotted] (Fq3l)
			(Fq5)
				edge[edgetower] (Fq25)
				edge[edgecomp] (dot4)
				edge[edgecomp] (dot6)
			(Fq25)
				edge[edgetower, dotted] (Fq5l)
			(Fql)
				edge[edgetower, blue] (Fql2)
				edge[edgecomp] (dot6)
			(Fql2)
				edge[edgetower, blue, dotted] (Fqll)
			(dot3)
				edge[edgecomp] (Fq2)
				edge[edgecomp] (Fq5)
			(dot5)
				edge[edgecomp] (Fq3)
				edge[edgecomp] (Fql);
		%---------------------
		\draw [xshift = 4.5cm] node [right, text width = 2.5cm, 
			rounded corners, fill=red!10, inner sep = 1mm]{
				\small
				\mbox{$\displaystyle\F_q^{(\ell)} = \bigcup_{i\ge 0}\F_{q^{\ell^i}},$}\\
                                \smallskip
				\mbox{$\displaystyle\bar{\F}_q \cong \bigotimes_{\ell\text{ prime}} \F_q^{(\ell)}$}
			};
		\draw[xshift = -3.8cm] node [left, rounded corners, fill=red!10, inner sep = 1mm]{
				\small $q$ and $\ell$ fixed
			};
	\end{tikzpicture}
\end{frame}
%###################################################

\begin{frame}
	
	\begin{tikzpicture}
		\node [rectangle, line width=0.7pt, draw=red!50] (abstract) { \hspace*{0.03\textwidth}%
		    \begin{minipage}{0.92\textwidth}
			    \vspace*{3mm}
			    \begin{itemize}
				    \item Efficient construction of towers
				    \item Efficient computation of embeddings
			    \end{itemize}
		    \end{minipage}\hspace*{0.03\textwidth}
		};%
		\node[right=0.03\textwidth,
		rectangle,
		line width=0.5pt,
		rounded corners=2pt,
		draw=red,
		fill=blue!20,
		inner xsep=3pt, inner ysep=3pt] at (abstract.north west) {\it\bfseries Our interest};
	\end{tikzpicture}
	
	\vspace*{3mm}
	\textbf{Goals}:
	\begin{itemize}
		\item Describing embeddings:
		\begin{itemize}
			\item Replace linear algebra by polynomial arithmetic.
		\end{itemize}
		\item Computing embeddings:
		\begin{itemize}
			\item Quasi-linear time and memory in the degree of the	extension.
		\end{itemize}
	\end{itemize}
	
	\vspace*{3mm}
	\textbf{Application examples}:
	\begin{itemize}
		\item In general: finite field arithmetic.
		\item Computing isogenies between elliptic curves, L. De Feo, 2011.
		\item Point-counting in genus 2, Gaudry and Schost, 2012.
	\end{itemize}
	
\end{frame}

%###################################################

{
\setbeamertemplate{itemize/enumerate body begin}{\fontsize{10pt}{12pt}\selectfont}
\setbeamertemplate{itemize/enumerate subbody begin}{\fontsize{9pt}{12pt}\selectfont\setlength{\leftmarginii}{2mm}}
\begin{frame}
	\frametitle{Previous works}

	\begin{itemize}
          \item Magma's implementation:
            \begin{itemize}
          \setlength{\itemindent}{0pt}
            \item Bosma, Cannon, Steel, \textit{Lattices of compatibly embedded finite fields}, 1997.
            \item E. Rains, \textit{Efficient computation of isomorphisms between finite fields}, 199x.
            \end{itemize}
		\item Artin-Schreier towers, $\ell=p$:
		\begin{itemize}
			\item D.~G. Cantor, \textit{On arithmetical algorithms over finite fields}, 1989.
			\item J.-M. Couveignes, \textit{Isomorphisms between {A}rtin-{S}chreier towers}, 2000.
			\item De Feo and Schost, \textit{Fast arithmetics in {A}rtin-{S}chreier towers over finite fields}, 2012.
		\end{itemize}
		\item $\ell$-adic towers, $\ell = 2$:
		\begin{itemize}
			\item Doliskani and Schost, \textit{A note on computations in degree $2^k$-extensions of finite fields}, 2012.
		\end{itemize}
		\item Inspiration behind this work:
		\begin{itemize}
			\item De Smit and Lenstra, \textit{Standard models for finite fields}, 2008.
			\item Couveignes and Lercier, \textit{Fast construction of irreducible polynomials over finite fields}, 2011.
			\item V. Shoup, \textit{Fast construction of irreducible polynomials over finite fields}, 1994.
		\end{itemize}
	\end{itemize}
\end{frame}
}
%###################################################
\begin{frame}
  \frametitle{Univariate vs. Multivariate}

  \begin{tikzpicture}
    \begin{scope}[node distance=2cm]
      \node (F0) {$\F_q$};
      \node[above of = F0] (F1) {$\K_1$};
      \node[above of = F1] (F2) {$\K_2$};
      \node[above of = F2] (Fi) {$\K_i$};
      \draw (F0) -- node[auto]{\small$\ell$} (F1) -- node[auto]{\small$\ell$} (F2);
      \draw[dashed] (F2) -- (Fi);
      \visible<2>{
        \path (F2) to[bend right] node (F2i){} (Fi);
        \draw[dashed,color=black!30,bend right,every node/.style={auto,swap,color=black}]
        (F0) edge node {$T_1$} (F1)
        (F1) edge node {$T_2$} (F2)
        (F2) edge (F2i)
        (F2i) edge (Fi);
      }
      \visible<3>{
        \draw[dashed,color=black!30,auto,swap,pos=0.8]
        (F0) edge[bend right=30] node[color=black] {$Q_1$} (F1)
        (F0) edge[bend right=50] node[color=black] {$Q_2$} (F2)
        (F0) edge[bend right=70] node[color=black] {$Q_i$} (Fi);
      }
    \end{scope}
    %%%%%%%%%%%%%%%%%%%%%%% 
    \uncover<2->{
      \node[right=3cm of F1] (T) {
        $\F_q[X_1,X_2,\dots,X_i] /
        \left|\begin{aligned}
            T_i(X_i, \dots, X_2, X_1)\\
            \vdots\qquad\\
            T_2(X_2, X_1)\\
            T_1(X_1)
          \end{aligned}\right.
        $
      };
    }
    \visible<2>{
      \draw[->] (Fi) .. controls +(0.8cm,0cm) .. +(1.6cm,-2cm)
      node[pos=1,auto,sloped] {$\sim$}
      .. controls +(2.4cm,-4cm) .. (T)  ;
    }
    \uncover<3->{
      \node[above = 2cm of T] (Q) {
        $\F_q[X_i] / Q_i(X_i)$
      };
    }
    \visible<3->{
      \draw[<->,dashed,color=black!50] 
      (Q) edge node {Embedding evaluation} (T);
      \draw[->] (Fi) to[bend left=10] node[auto,sloped] {$\sim$} (Q);
    }
  \end{tikzpicture}
\end{frame}

%###################################################

\begin{frame}
	\frametitle{Summary of Main Results}
	
	\begin{block}{Objective}
		\footnotesize
		\begin{itemize}
			\item $q$ and $\ell$ fixed,
                        \item \alert{$\tildO(\ell^i)$} operations in $\F_q$,
                        \item limit additional factors in $\ell$ and $q$ as much as possible.
		\end{itemize}
	\end{block}

	\resizebox{0.93\textwidth}{!}{
          \footnotesize
		\begin{minipage}{\textwidth}
			\[
			\begin{array}{c|cccc}
				\text{Condition} & \text{\bf Initialization} & \mathbf{Q_i, T_i} & \text{\bf Embedding eval.}\\
				\hline \hline
				q = 1 \bmod \ell & O(1)  & O(\ell^i) & O(\ell^i) \\
				q = -1 \bmod \ell & O(1) & O(\ell^i) & O(\Mult(\ell^i) \log(\ell^i)) \\
				- & O (\ell^2) & O(\Mult(\ell^{i+1})\Mult(\ell)\log(\ell^i)^2) & O( \Mult(\ell^{i+1})\Mult(\ell)\log(\ell^i))\\
				4\ell \le q^{1 / 4} & \tildO(\ell^3) \text{~(bit)}  & O(\Mult(\ell^i)\log(\ell^i)) & O(\Mult(\ell^i) \log(\ell^i)) \\
				4\ell \le q^{1 / 4} & \tildO(\Mult(\ell)) & O(\Mult(\ell^i) \log(\ell^i)) & O(\Mult(\ell^i) \log(\ell^i)) \vspace{-0.5cm}
			\end{array}
			\]
		\end{minipage}
	}
\end{frame}

%**********************************************************
%**********************************************************

\section{Quasi-cyclotomic towers}

\begin{frame}
	\frametitle{\insertsection 
	\\{\small\normalfont\footnotesize \color{black} 
	(inspired by Shoup, De Smit and Lenstra)}}
	
    \begin{columns}
      \begin{column}{0.65\textwidth}
	\begin{center}
		\begin{tikzpicture}[thick, node distance=2cm, inner sep = 2pt]
			\node[blue] (Q) {$\F_q$};
			\node (Q0) [above left = 1cm of Q]{$\F_q(\zeta_\ell)$};
			\node[blue] (K1) [above right of=Q]{$\F_q(x_1)$};
			\node (Q1) [above right of=Q0]{$\F_q(\zeta_{\ell^2})$};
			\node[blue] (K2)[above right of=K1]{$\F_q(x_2)$};
			\node (Q2) [above right of=Q1]{$\F_q(\zeta_{\ell^3})$};
			\node[blue] (Koo)[above right of=K2]{$\quad \F_q(x_i)$};
			\node (Qoo) [above right of=Q2]{$\quad\F_q(\zeta_{\ell^i})$};
			% --------------------------
			\draw (Q) edge[red] node[auto]{$r$} (Q0)
                        edge[blue] node[auto,swap]{$\ell$} (K1)
                        (K1) edge[red] node[red, auto] {$r$} (Q1)
                        edge[blue] node[auto,swap]{$\ell$} (K2)
                        (Q1) edge node[auto]{$\ell$} (Q0)
                        (Q2) edge[red] node[red, auto, swap]{$r$} (K2)
                        edge node[auto]{$\ell$} (Q1)
                        edge[dotted] (Qoo)
                        (Koo) edge[blue, dotted] (K2)
                        edge[red] node[red, auto]{$r$} (Qoo);
			% --------------------------
                        \uncover<1>{
                        \node (cyclo) [above left = 2mm of Q2, fill = red!10, rounded corners, font=\small]{ Cyclotomic fields };
                        \draw[dashed, ->, color = black!30] (cyclo) edge[bend right] (Q0)
                         edge[bend right] (Q1)
                         edge[bend left] (Q2)
                         edge[bend left] (Qoo);}
			% --------------------------
                       \uncover<2>{
                         \draw[dashed,bend right=40,color = black!30]
                         (K1) edge node[auto,swap,color=black]{$T_2$} (K2)
                         (Q) edge node[auto,swap,color=black]{$Q_i$} (Koo);
                       }
		\end{tikzpicture}
	\end{center}
      \end{column}
      \begin{column}{0.35\textwidth}
        \begin{block}{}
          \begin{itemize}
          \item<1-> $r\mid(\ell-1)$;
          \item<1-> $x_i = \Tr_{\K_i / \F_{q^{\ell^i}}}(\zeta_{\ell^i})$;
          \item<2-> Both $T_i$ and $Q_i$
            can be computed by
            resultants.
          \end{itemize}
        \end{block}
      \end{column}
    \end{columns}
\end{frame}

%###################################################

\begin{frame}
  \frametitle{\insertsection }

  \begin{exampleblock}{Generic algorithm}
    \begin{itemize}
    \item Perform all computations in the \emph{cyclotomic tower};
    \item Construction and embedding evaluation: penalty only \alert{$\tildO(\ell^2)$}.
    \end{itemize}
  \end{exampleblock}

  \begin{overlayarea}{\textwidth}{5cm}
    \begin{onlyenv}<1>
      \begin{exampleblock}{Trivial case: $\ell \mid (q-1) \Leftrightarrow r=1$}
        Kummer extensions
        \[\alert{Q_i=X_i^{\ell^i}-y_0} \qquad\text{and}\qquad \alert{T_i=X_{i}^\ell-X_{i-1}}\]
        Embeddings are trivial.
      \end{exampleblock}
    \end{onlyenv}	

    \begin{onlyenv}<2>
      \begin{exampleblock}{Special case: $\ell \mid (q + 1) \Leftrightarrow r = 2$}
        By direct resultant computation
        \[\alert{Q_i(X_i) = Y^{\ell^i} + Y^{-\ell^i} - x_0 \mod Y^2-X_iY+1}\]

        \vspace*{-3mm}
        \begin{itemize}
        \item Similar form for $T_i$.
        \item $Q_i$ can be computed in $O(\Mult(\ell^i))$; a better algorithm \textcolor{blue}{later}.
        \item Embeddings: \textcolor{blue}{later}.
        \end{itemize}
      \end{exampleblock}
    \end{onlyenv}	
  \end{overlayarea}
\end{frame}


%**********************************************************
%**********************************************************

\section{Towers from irreducible fibers}


\begin{frame}
  \frametitle{\insertsection\ (Couveignes and Lercier, 2011)}

  \begin{center}
    \large
    $\ell\mid(q-1)$, consider the map \alert{$\phi:x \mapsto x^\ell$}
  \end{center}

  \begin{columns}
    \begin{column}{0.6\textwidth}
      \begin{center}
        \begin{tikzpicture}
          \begin{uncoverenv}<1->
            \draw (-60:-1) circle (2);
            \path (-60:-1) -- +(-1.4,0.4) node {\Large\alert{$\F_q^\ast$}};
            \draw (0:0) node {$1$};
          \end{uncoverenv}
          \begin{uncoverenv}<2->
            \foreach \i in {1,...,4} {
              \draw (50*\i:0.2) -- (50*\i:0.6);
              \draw (50*\i:0.9) node {$\zeta_\ell^\i$};
            }
          \end{uncoverenv}
          \begin{uncoverenv}<3->
            \foreach \i in {1,3,4} {
              \foreach \j in {0,...,4} {
                \draw[dotted] (50*\i:1.2) -- (50*\i+8*\j-20:1.6);
              }
            }
            \foreach \j in {0,...,4} {
              \draw (100:1.2) -- (100+8*\j-20:1.8);
            }
            \foreach \k in {0,...,4} {
              \draw (104:1.8) -- (104+8*\k-20:2.3);
            }
            \draw (108:2.5) node {$y_0$};
          \end{uncoverenv}
          \begin{uncoverenv}<4->
            \draw (-60:-1.5) circle (2.8);
            \path (-60:-1) -- +(-2.1,1.4) node (Fl) {\Large\alert{$\F_{q^\ell}^\ast$}};
            \foreach\k in {0,...,4} {
              \draw (108:2.7) -- (108+8*\k-20:3.5);
            }
            \draw (112:3.7) node {$y_1$};
          \end{uncoverenv}
          \begin{uncoverenv}<5->
            \path (-60:-1) -- +(-3.6,2.8) node (Fbar) {\Large\alert{$\bar{\F}_q^\ast$}};
            \path (Fl) -- (Fbar) node[pos=0.5,sloped] {\Large\alert{$\dots$}};
            \foreach\k in {0,...,4} {
              \draw[dashed] (112:3.9) -- (112+8*\k-20:4.6);
            }
          \end{uncoverenv}
        \end{tikzpicture}
      \end{center}
    \end{column}
    \begin{column}{0.4\textwidth}
      \begin{itemize}
      \item<3-> $\phi|_{\F_q^\ast}$ not surjective;
      \item<4-> $\phi : \mathbb{G}_m \to \mathbb{G}_m$ surjective;
      \item<5-> Starting from $y_0$, every $\phi^{-1}y_i$ is an
        irreducible set of cardinality $\ell$.
      \end{itemize}
    \end{column}
  \end{columns}
\end{frame}

%###################################################

\begin{frame}
  \frametitle{\emph{Chebyshev} case: $\ell\mid(q+1)$}
  
  \begin{center}
    \large
    Consider the map \alert{$\phi:x \mapsto x^\ell$}
  \end{center}

  \begin{columns}
    \begin{column}{0.45\textwidth}
      \begin{tikzpicture}[shorten >=2pt]
        \uncover<1-2>{\draw (0:0) node {\alert{\large$\F_q^\ast$}};}
        \foreach \i in {0,...,7} {
          \pgfmathparse{int(2+!mod(\i,4))}
          \let\stop\pgfmathresult

          \uncover<1-\stop>{\fill (\i*45:1) circle (1pt);}

          \uncover<2-\stop>{
            \foreach \j in {0,...,3} {
              \draw[->] (\i*45-16.875+\j*11.25:2) -- (\i*45:1);
            }
          }
        }

        \foreach \i in {0,...,3} {
          \pgfmathparse{int(2+!mod(\i,2))}
          \let\stop\pgfmathresult

          \uncover<1-2>{
            \draw[->,bend right=50] (\i*90+45:1) to (\i*90+225:1);
          }
          \uncover<1-\stop>{
            \draw[->,in=\i*90+180-30,out=\i*90+180+30,loop] (\i*90:1) to ();
          }
        }

        \uncover<2>{
          \draw[dashed,red] (0:0) circle (1.5);
        }
        \uncover<2>{
          \draw (-90:2.5) node {\large\alert{$\F_{q^2}^\ast$}};
        }
        \uncover<3->{
          \draw[dashed,red] (0:0) circle [x radius=2.5, y radius=1];
          \draw (-30:2.5) node {\large\alert{$\mathbb{T}_2$}};
        }
      \end{tikzpicture}
    \end{column}
    \begin{column}{0.5\textwidth}
      \begin{itemize}
      \item<1-> $\phi|_{\F_q^\ast}$ bijective;
      \item<2-> $\phi|_{\F_{q^2}^\ast}$ non surjective;
      \item<3-> $\mathbb{T}_2\subset\F_{q^2}^\ast$ \emph{algebraic torus} of
        cardinality $q+1$.
      \end{itemize}
    \end{column}
  \end{columns}

  \begin{block}{}<3>
    \[\mathbb{T}_n(k) \cong
    \{\alpha\in L^\ast \;|\; \Norm_{L/F}(\alpha) = 1 \text{ for all } k\subset F\subsetneq L \}.\]
  \end{block}
\end{frame}

%###################################################

\begin{frame}
	\frametitle{Towers from algebraic tori (Pell conics)}

        \begin{itemize}
        \item By Weil descent, $\mathbb{T}_2$ is isomorphic to a
          \emph{Pell conic};
        \item Multiplication in $\bar{\F}_q$ induces a group law on
          the points.
        \end{itemize}
        
	\tikzset{
		point style/.style={circle, inner sep=0.9pt, fill = blue},
		extended line/.style={shorten >= -5mm, shorten <= -5mm}
	}

	\begin{center}
		\begin{tikzpicture}[scale = 0.9, every node/.style={scale=0.9}]
			\draw[name path = pellconic] (-1.5, 2.25) parabola bend (0, 0) (2.1, 4.41);
			\draw[style=help lines] (-1.7, 0) grid (2.2,4.5);
			%-----------
			\coordinate (N) at (1.3, 1.69);
			\coordinate (P) at (1.7, 2.89);
			\coordinate (Q) at (-1.3, 1.69);
			\coordinate (PQtemp) at ($(Q) + (N) - (P)$);
			%-----------
			\draw[extended line] (P)--(Q);
			\path[name path = pline] (PQtemp)--(N);
			\path [name intersections={of = pellconic and pline}];
			\coordinate (PQ) at (intersection-1);
			\draw [extended line] (N) -- (PQ);
			%-----------	
			\node[point style, fill = red, label=below right:$N$] at (N) {};
			\node[point style, label=below right:$P$] at (P) {};
			\node[point style, label=below left:$Q$] at (Q) {};
			\node[point style, label=below left:$P + Q$] at (PQ) {};

			\node[right = 4mm of current bounding box, text width = 7cm, fill = red!10, rounded corners, font = \small] (explain) {
				\textcolor{blue}{\it Pell conic:}
				\begin{equation*}
					\label{eq:Pell}
					C \;:\; x^2 - \Delta y^2 = 4
				\end{equation*}
				
				\vspace*{1cm}
				\textcolor{red}{Addition:} For $P=(x_1,y_1)$ and $Q=(x_2,y_2)$,
				\begin{equation*}
					P\oplus Q = \left(\frac{x_1x_2 + \Delta y_1y_2}{2},\; \frac{x_1y_2 + x_2y_1}{2}\right)
				\end{equation*}
			};
		\end{tikzpicture}	
	\end{center}
\end{frame}

%###################################################

\begin{frame}
	\frametitle{Towers from algebraic tori}

        \begin{itemize}
        \item $\mathbb{T}_2 \to$ Pell conic $C$,
        \item multiplication in $\F_{q^2} \to$ addition in $C$,
        \item $\ell$-th power $\to$ scalar multiplication $[\ell]$.
        \end{itemize}
	
	\begin{lemma}
          The abscissa of $[n]P$ is given by $C_n(x_1)$, where
          $C_n\in\Z[X]$ is the $n$-th \textcolor{blue}{Chebyshev
            polynomial}.
	\end{lemma}
	
	\begin{theorem}
          Let $P$ be a point not in $\ell C$, then we can compute
          \begin{equation*}
            Q_i(X_i) = C_{\ell^i}(X_i) - x_P
            \quad\text{and}\quad
            \alert{T_i(X_i) = C_{\ell}(X_i) - X_{i-1}}
          \end{equation*}
          using $O(\ell^i)$ operations.
        \end{theorem}
\end{frame}

%###################################################

\begin{frame}[fragile]
	\frametitle{Towers from elliptic curves}

        \begin{itemize}
        \item \alert{Problem 1:} there is essentially one conic; we
          would like to have more group choices, \alert{elliptic
            curves} are an option.
        \item \alert{Problem 2:} $\ell$-multiplication on elliptic
          curves is a degree $\ell^2$ map; we must consider
          \alert{separable isogenies} instead.
        \end{itemize}
	
	\newcommand{\EC}[1]{{\color{blue} \ensuremath{E_{#1}}}}
	\newcommand{\IS}[1]{{\color{red} \ensuremath{\phi_{#1}}}}
	
	\begin{equation*}
		E_0 \;:\; y^2 = x^3 + ax + b, \quad a, b \in\F_q,  \quad \ell \nmid (q-1), \quad \ell \mid \#E_0(\F_q)
	\end{equation*}

	\begin{center}
		\begin{tikzpicture}
			\def\n{4}
			\foreach \i in {0,...,\n} {
				\pgfmathparse{360/(\n+1)*(\i-1/2) - 90}
				\let\angle\pgfmathresult
				\draw (\angle:1) node (E\i) {\EC{\i}};
				% \draw (\angle:0.4) node {\scriptsize$H_\i$};
				% \draw (\angle:0.7) node[rotate=\angle] {\scriptsize$\subset$};
			}
			\foreach \i in {0,...,\n} {
				\pgfmathparse{int(mod(\i+1, \n+1))}
				\let\j\pgfmathresult
				\draw (E\i) edge[->,bend right=18] node[auto, swap] {\scriptsize \IS{\i}} (E\j);
			}

			\node[left = -2.5mm of current bounding box.west] (explain) {
			\begin{minipage}{0.75\textwidth}
				Under these assumptions, isogenies form a cycle
                                \[\IS{i}:\EC{i} \to \EC{i + 1}.\]

				\textbf{\color{orange} Lemma }{\it
				   $\EC{n} \cong \EC{0}$ for some $n \in O(\sqrt{q}\log (q))$.}
			\end{minipage}
			};
			\begin{pgfonlayer}{background}
				\path[fill=yellow!10, rounded corners = 5pt, draw = blue!20] (current bounding box.south west) rectangle
				(current bounding box.north east);
			\end{pgfonlayer}
		\end{tikzpicture}
	\end{center}
\end{frame}

%###################################################

\begin{frame}
	\frametitle{Towers from elliptic curves}
	
        \begin{lemma}[Couveignes and Lercier, 2011]
          Let $P \not\in \ell E_i$, and $\psi=\phi_{i-1}\circ\phi_{i-2}\circ\cdots\circ\phi_{j}$,\\
          then $\psi^{-1}(P)$ is irreducible of cardinality $\ell^{i-j}$.
        \end{lemma}
	\vspace*{5mm}
	\begin{exampleblock}{V\'elu's formulas}
		\begin{equation*}
		  \begin{array}{crcl}
		    \phi_i: & E_i &\longrightarrow & E_{i+1},\\
		    & (x,y) &\longmapsto & \left(\frac{f_i(x)}{g_i(x)}, y\left(\frac{f_i(x)}{g_i(x)}\right)'\right),
		  \end{array}
		\end{equation*}
	\end{exampleblock}
	
	\begin{block}{The $\ell$-adic tower}
		\begin{align*}
		  T_1 &= f_{-1}(X_1) - \eta g_{-1}(X_1),\\ 
		  \alert{T_i} &\alert{= f_{-i}(X_i) - X_{i-1} g_{-i}(X_i)}.
		\end{align*}
	\end{block}
\end{frame}

%**********************************************************
%**********************************************************

\section{Evaluating embeddings}

\begin{frame}
  \frametitle{\insertsection}
  
  \begin{block}{Observation}
    In all previous cases, from the form of $T_i$ we deduce
    \[X_{i-1} = f(X_i)/g(X_i)\]
    for some $f$ and $g$.  
    Going from multivariate to univariate is
    \[\sum a_j \alert{X_{i-1}^{\alpha_j}} X_i^{\beta_j} \mapsto
    \sum a_j \alert{\frac{f(X_i)^{\alpha_j}}{g(X_i)^{\alpha_j}}} X_i^{\beta_j} \]
  \end{block}
  
  \begin{definition}
    Let $P \in \F_q[X,Y]$ and $n \in \N$, with $\deg(P,X)< n$. Define
    $$P[f,g,n] = g^{n-1} P\left (\frac fg, Y\right) \in \F_q[Y].$$
  \end{definition}
\end{frame}

%###################################################

\begin{frame}
	\frametitle{Lifting: Multivariate $\to$ Univariate}
	
	\small
	\begin{center}
		\begin{tikzpicture}[every node/.style={scale=0.7}]
			\node [fill = red!10, draw = red!80]{
				\begin{minipage}{0.95\textwidth}
					\begin{algorithm}[H]
						\caption{Compose}
						\label{alg:compose}
						\begin{algorithmic}[1]
							\REQUIRE $P\in \F_q[X,Y]$, $f,g\in \F_q[Y]$, $n\in\N$
							\IF {$n = 1$} 
								\RETURN $P$
							\ELSE
								\STATE $m \la \lceil n/2\rceil$
								\STATE Let $P_0,P_1$ be such that $P = P_0 + X^mP_1$
								\STATE $Q_0 \la$ Compose($P_0, f, g, m$)
								\STATE $Q_1 \la$ Compose($P_1, f, g, n-m$)
								\STATE $Q \la Q_0g^{n-m} + Q_1f^m$  \label{alg:compose:res}
								\RETURN $Q$
							\ENDIF
						\end{algorithmic}
					\end{algorithm}
				\end{minipage}
			};
		\end{tikzpicture}
	\end{center}
	
	\begin{theorem}
	  \label{th:compose}
	  Algorithm~\ref{alg:compose} computes $Q=P[f,g,n]$ using $O(\Mult(\ell
	  n)\log(n))$ operations in $\F_q$.
	\end{theorem}
\end{frame}

%###################################################

\begin{frame}
	\frametitle{Pushing: Univariate $\to$ Multivariate}
	
	\small
	\begin{center}
		\begin{tikzpicture}[every node/.style={scale=0.7}]
			\node [fill = red!10, draw = red!80]{
				\begin{minipage}{0.95\textwidth}
					\begin{algorithm}[H]
					  \caption{Decompose}
					  \label{alg:decompose}
					  \begin{algorithmic}[1]
					    \REQUIRE $Q,f,g,h\in \F_q[Y]$, $n \in \N$ 
					    \IF {$n=1$} 
						    \RETURN $Q$
					    \ELSE 
						    \STATE $m \la \lceil n/2 \rceil$ 
						    \STATE $u \la 1/ g^{n-m}\bmod f^m$ \label{alg:decompose:xgcd} 
						    \STATE $Q_0 \la Q u \bmod f^m$ 
						    \STATE $Q_1 \la (Q-Q_0 g^{n-m}) {\rm~div~} f^m$ 
						    \STATE $P_0 \la$ Decompose($Q_0, f, g, h, m$) 
						    \STATE $P_1 \la$ Decompose($Q_1, f, g, h, n-m$) 
						    \RETURN $P_0 + X^m P_1$ 
					    \ENDIF
					  \end{algorithmic}
					\end{algorithm}
				\end{minipage}
			};
		\end{tikzpicture}
	\end{center}
	
	\begin{theorem}
	  Algorithm~\ref{alg:decompose} computes a polynomial $P\in \F_q[X,Y]$
	  such that $Q=P[f,g,n]$ using
	  $O(\Mult(\ell n)\log(n))$ operations in $\F_q$.
	\end{theorem}
\end{frame}

%**********************************************************
%**********************************************************

\section{Implementation}

\begin{frame}
	\frametitle{\insertsection}
	
	\begin{center}
		\includegraphics[width=9cm]{../creat}\\\vspace*{1mm}
		\parbox{7cm}{\scriptsize Times for building $3$-adic towers on top of $\F_2$ (left)
				and $\F_5$ (right), in Magma (first three lines) and using our code.}
	\end{center}
	
	\begin{itemize}
		\small
		\item Intel Xeon E5620 clocked at 2.4 GHz, using Sage 5.5 and Magma 2.18.12
		\item Source code at \url{https://github.com/defeo/towers}.
	\end{itemize}
\end{frame}

%**********************************************************
%**********************************************************

%\section{Conclusions}
%
%\begin{frame}
%	\frametitle{\insertsection}
%	
%	\begin{itemize}
%		\item We have introduced techniques for
%		\begin{itemize}
%			\item creating $\ell$-adic towers,
%			\item computing embeddings
%		\end{itemize}
%		based on polynomial arithmetic.
%		\item Quasi-linear time and memory in the extension degree.
%		\item Despite its simplicity, our code consistently outperforms Magma on creation time.
%		\begin{itemize}
%			\item Lift and push operations take	essentially no time in Magma.
%		\end{itemize}
%	\end{itemize}	
%\end{frame}








\end{document}









