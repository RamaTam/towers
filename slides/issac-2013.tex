
\documentclass[10pt]{beamer}


\usepackage[english]{babel}
\usepackage{amsmath}               
\usepackage{amsfonts}              
\usepackage{amsthm}                
\usepackage{amssymb}
\usepackage{graphicx}
\usepackage{color}
\usepackage{../mysymbols}
\usepackage{tikz}
\usefonttheme{serif}
\usetikzlibrary{positioning}
\usetikzlibrary{calc}
\usetikzlibrary{decorations.pathreplacing}
\usetikzlibrary{matrix}
\usetikzlibrary{intersections}
\usetikzlibrary{backgrounds}
\usetikzlibrary{fit}
\usepackage{algorithmic}
\usepackage{algorithm}




\usetheme{Boadilla}
\usecolortheme{lily}
\setbeamertemplate{navigation symbols}{}
\setbeamertemplate{footline}{}
\setbeamercovered{transparent}
\setbeamertemplate{headline}
{
  \leavevmode%
  \hbox{%
  \begin{beamercolorbox}[wd=.4\paperwidth,ht=2.25ex,dp=1ex,center]{author in head/foot}%
    \usebeamerfont{author in head/foot}\insertsection
  \end{beamercolorbox}%
  \begin{beamercolorbox}[wd=.6\paperwidth,ht=2.25ex,dp=1ex,right]{title in head/foot}%
    \insertframenumber{} / \inserttotalframenumber\hspace*{1ex}\hspace*{2mm}
  \end{beamercolorbox}}%
  \vskip0pt%
}
\setbeamertemplate{frametitle}{
	\begin{centering}
		\textbf{\small \insertframetitle}
		\par
	\end{centering}
}


\newcommand{\Cyc}{\Phi}  % Cyclotomic polynomial
\newcommand{\bb}{\mathbf{B}}  
\newcommand{\uu}{\mathbf{U}}  % bases for lift and push
\DeclareMathOperator{\Res}{Res}


\newtheorem{proposition}[theorem]{Proposition}

\renewcommand{\algorithmicrequire}{\textbf{Input:}}
\renewcommand{\algorithmicensure}{\textbf{Output:}}




\title
{\bf Fast algorithms for {\Large$\mathbf{\ell}$}-adic\\ towers over finite fields}


\author{
	Luca De Feo\inst{1},\and\hspace*{-3mm} 
	Javad Doliskani\inst{2},\and\hspace*{-3mm} 
	\'Eric Schost\inst{2}
}

\institute{
  \inst{1}Laboratoire PRiSM, Universit\'e de Versailles \and
  \vspace*{-2mm}
  \inst{2}Department of Computer Science, Western University
}

\subject{Computer Algebra}

\date{ISSAC 2013}

\pgfdeclareimage[height = 5mm]{westernlogo}{Western_Logo_CMYK}
\pgfdeclareimage[height = 5mm]{versailleslogo}{uvsq-logo}

\begin{document}

\setbeamertemplate{itemize subitem}[triangle]
\setbeamercolor{itemize subitem}{fg=red}

\begin{frame}
	\thispagestyle{empty}
	\titlepage
	\begin{flushright}
		\pgfuseimage{westernlogo}\hspace*{1mm}\pgfuseimage{versailleslogo}
	\end{flushright}
\end{frame}

%**********************************************************
%**********************************************************

\section{Introduction}

\begin{frame}
	\frametitle{Problem statement}
	
	\tikzset{
		dotstyle/.style={circle, inner sep = 1.2pt, outer sep = 4pt, fill = gray},
		edgetower/.style={thick},
		edgecomp/.style={thick, lightgray}
	}
	\begin{tikzpicture} [scale = 0.8] %, every node/.style={inner sep = 2pt, scale=0.8}]
		\coordinate (T2) at (-2, 0.5);
		\node (Fq) at (0, 0) {$\F_q$};
		\node (Fq2) at ($(Fq) + (T2)$) {$\F_{q^2}$};
		\node (Fq4) at ($(Fq2) + (T2)$) {$\F_{q^4}$};
		\node (Fq2l) at ($(Fq4) + (T2)$) {$\F_q^{(2)}$};
		%---------------------
		\coordinate (T3) at (-0.7, 2);
		\node (Fq3) at ($(Fq) + (T3)$) {$\F_{q^3}$};
		\node (Fq9) at ($(Fq3) + (T3)$) {$\F_{q^9}$};
		\node (Fq3l) at ($(Fq9) + (T3)$) {$\F_q^{(3)}$};
		%---------------------
		\coordinate (T5) at (0.7, 2);
		\node (Fq5) at ($(Fq) + (T5)$) {$\F_{q^5}$};
		\node (Fq25) at ($(Fq5) + (T5)$) {$\F_{q^{25}}$};
		\node (Fq5l) at ($(Fq25) + (T5)$) {$\F_q^{(5)}$};
		%---------------------
		\coordinate (Tl) at (2, 1);
		\node[blue] (Fql) at ($(Fq) + (Tl)$) {$\F_{q^\ell}$};
		\node[blue] (Fql2) at ($(Fql) + (Tl)$) {$\F_{q^{\ell^2}}$};
		\node[blue] (Fqll) at ($(Fql2) + (Tl)$) {$\F_q^{(\ell)}$};
		%---------------------
		\node[dotstyle] (dot1) at ($(Fq2) + (Fq3) - (Fq)$) {};
		\node[dotstyle] (dot2) at ($(Fq4) + (dot1) - (Fq2)$) {};
		\node[dotstyle] (dot3) at ($(Fq2) + (Fq5) - (Fq)$) {};
		\node[dotstyle] (dot4) at ($(Fq3) + (Fq5) - (Fq)$) {};
		\node[dotstyle] (dot5) at ($(Fq3) + (Fql) - (Fq)$) {};
		\node[dotstyle] (dot6) at ($(Fq5) + (Fql) - (Fq)$) {};
		%---------------------
		\draw 
			(Fq) 
				edge[edgetower] (Fq2)
				edge[edgetower] (Fq3)
				edge[edgetower] (Fq5)
				edge[edgetower, blue] (Fql)
			(Fq2)
				edge[edgetower] (Fq4)
				edge[edgecomp] (dot1)
			(Fq4)
				edge[edgetower, dotted] (Fq2l)
				edge[edgecomp] (dot2)
			(dot1)
				edge[edgecomp] (dot2)
			(Fq3)
				edge[edgetower] (Fq9)
				edge[edgecomp] (dot1)
				edge[edgecomp] (dot4)
			(Fq9)
				edge[edgetower, dotted] (Fq3l)
			(Fq5)
				edge[edgetower] (Fq25)
				edge[edgecomp] (dot4)
				edge[edgecomp] (dot6)
			(Fq25)
				edge[edgetower, dotted] (Fq5l)
			(Fql)
				edge[edgetower, blue] (Fql2)
				edge[edgecomp] (dot6)
			(Fql2)
				edge[edgetower, blue, dotted] (Fqll)
			(dot3)
				edge[edgecomp] (Fq2)
				edge[edgecomp] (Fq5)
			(dot5)
				edge[edgecomp] (Fq3)
				edge[edgecomp] (Fql);
		%---------------------
		\draw [xshift = 4.5cm] node [right, text width = 2.5cm, 
			rounded corners, fill=red!10, inner sep = 1mm]{
				\small
				\[\F_q^{(\ell)} = \bigcup_{i\ge 0}\F_{q^{\ell^i}},\]
				\[\bar{\F}_q \cong \bigotimes_{\ell\text{ prime}} \F_q^{(\ell)}\]
			};
		\draw[xshift = -3.8cm] node [left, rounded corners, fill=red!10, inner sep = 1mm]{
				\small $q$ and $\ell$ fixed
			};
	\end{tikzpicture}
\end{frame}
%###################################################

\begin{frame}
	
	\begin{tikzpicture}
		\node [rectangle, line width=0.7pt, draw=red!50] (abstract) { \hspace*{0.03\textwidth}%
		    \begin{minipage}{0.92\textwidth}
			    \vspace*{3mm}
			    \begin{itemize}
				    \item Efficient construction of towers
				    \item Efficient computation of embeddings
			    \end{itemize}
		    \end{minipage}\hspace*{0.03\textwidth}
		};%
		\node[right=0.03\textwidth,
		rectangle,
		line width=0.5pt,
		rounded corners=2pt,
		draw=red,
		fill=blue!20,
		inner xsep=3pt, inner ysep=3pt] at (abstract.north west) {\it\bfseries Our interest};
	\end{tikzpicture}
	
	\vspace*{3mm}
	\textbf{Motivations}:
	\begin{itemize}
		\item Describing embeddings:
		\begin{itemize}
			\item Replace linear algebra by polynomial arithmetic.
		\end{itemize}
		\item Computing embeddings:
		\begin{itemize}
			\item Quasi-linear time and memory in the degree of the	extension.
		\end{itemize}
	\end{itemize}
	
	\vspace*{3mm}
	\textbf{Application examples}:
	\begin{itemize}
		\item In general: finite field arithmetic.
		\item Computing isogenies between ordinary elliptic curves, L. De Feo, 2011.
		\item Point-counting in genus 2, Gaudry and Schost, 2012.
	\end{itemize}
	
\end{frame}

%###################################################

\begin{frame}
	\frametitle{Previous works}
	
	\begin{itemize}
		\item Artin-Schreier towers, $\ell=p$:
		\begin{itemize}
			\item D.~G. Cantor, \textit{On arithmetical algorithms over finite fields}, 1989.
			\item J.-M. Couveignes, \textit{Isomorphisms between {A}rtin-{S}chreier towers}, 2000.
			\item De Feo and Schost, \textit{Fast arithmetics in {A}rtin-{S}chreier towers over finite fields}, 2012.
		\end{itemize}
		\item $\ell$-adic towers, $\ell = 2$:
		\begin{itemize}
			\item Doliskani and Schost, \textit{A note on computations in degree $2^k$-extensions of finite fields}, 2012.
		\end{itemize}
		\item Inspiration behind this work:
		\begin{itemize}
			\item Lenstra and De Smit, \textit{Standard models for finite fields}, 2008.
			\item Couveignes and Lercier, \textit{Fast construction of irreducible polynomials over finite fields}, 2011.
			\item V. Shoup, \textit{Fast construction of irreducible polynomials over finite fields}, 1994.
		\end{itemize}
	\end{itemize}
\end{frame}

%###################################################

\begin{frame}
	\frametitle{Remarks}
	
	\begin{exampleblock}{}
		\begin{itemize}
			\item Multiplications and inversions are quasi-linear in the extension degree.
			\item The Embedding problem is reduced to a pair of consecutive levels.
		\end{itemize}
	\end{exampleblock}
	
	\begin{center}
		\begin{tikzpicture}
			\footnotesize
			\node (A) {$\F_{q^{\ell^{i}}}$};
			\node[below left=1cm of A] (B) {$\F_q[X_{i}]/ \langle Q_{i} \rangle$};
			\node[below right=1cm of A] (C) {$\F_q[X_{i-1},X_i]/\langle Q_{i-1}(X_{i-1}), T_i(X_{i-1},X_i)\rangle$};
			\node[below=1.5cm of B] (D) {$\uu_i = (1,x_{i},x_{i}^2,\ldots,x_{i}^{\ell^{i}-1})$};
			\node[below=1.5cm of C] (E) { %
				$\begin{aligned}
					\bb_{i} = ( & 1,\ldots,x_{i-1}^{\ell^{i-1}-1},\ldots, \\
					& x_i^{\ell-1},\ldots,x_{i-1}^{\ell^{i-1}-1}x_i^{\ell-1})
				\end{aligned}$ %
			};
			%
			\path[->] (A) edge node[above,sloped,inner sep=0.5pt] {$ \sim $} (B);
			\path[->] (A) edge node[above,sloped,inner sep=0.5pt] {$ \sim $} (C);
			\path[->] (B) edge node[fill = white] {has basis} (D);
			\path[->] (C) edge node[fill = white] {has basis} (E);
			\path[->] (D.north east) edge[out = 30, in = 150] node[above] {\color{blue} \it push} (E.north west);
			\path[->] (E.west) edge[out = 210, in = 330] node[below] {\color{blue} \it lift} (D.south east);
		\end{tikzpicture}
	\end{center}
\end{frame}

%###################################################

\begin{frame}
	\frametitle{Summary of Main Results}
	
	\resizebox{0.93\textwidth}{!}{
		\begin{minipage}{\textwidth}
			\[
			\begin{array}{c|cccc}
				\text{Condition} & \text{\bf Initialization} & \mathbf{Q_i, T_i} & \text{\bf Lift, push}\\
				\hline \hline
				q = 1 \bmod \ell & O_e(1)  & O(\ell^i) & O(\ell^i) \\
				q = -1 \bmod \ell & O_e(1) & O(\ell^i) & O(\Mult(\ell^i) \log(\ell^i)) \\
				- & O_e (\ell^2) & O(\Mult(\ell^{i+1})\Mult(\ell)\log(\ell^i)^2) & O( \Mult(\ell^{i+1})\Mult(\ell)\log(\ell^i))\\
				4\ell \le q^{1 / 4} & O\tilde{_e}(\ell^3) \text{~(bit)}  & O_e(\Mult(\ell^i)\log(\ell^i)) & O(\Mult(\ell^i) \log(\ell^i)) \\
				4\ell \le q^{1 / 4} & O\tilde{_e}(\Mult(\ell)) & O_e(\Mult(\ell^i) \log(\ell^i)) & O(\Mult(\ell^i) \log(\ell^i)) \vspace{-0.5cm}
			\end{array}
			\]
		\end{minipage}
	}
	
	\vspace*{1.5cm}
	\begin{block}{Notations}
		\footnotesize
		\begin{itemize}
			\item $\Mult$: polynomial multiplication
			\item $O_e(\ )$: probabilistic with expected running time
			\item $O\tilde{_e}(\ )$: additional omission of logarithmic factors.
			\item bit: bit complexity.
		\end{itemize}
	\end{block}
\end{frame}

%**********************************************************
%**********************************************************

\section{Quasi-cyclotomic towers}

\begin{frame}
	\frametitle{\insertsection 
	\\{\small\normalfont\footnotesize \color{black} 
	(inspired by Shoup, Lenstra and De Smit, Couveignes and Lercier)}}
	
	\begin{center}
		\begin{tikzpicture}[thick, node distance=2cm, inner sep = 2pt]
			\node[blue] (Q) {$\F_q$};
			\node (Q0) [above left = 1cm of Q]{$\K_0=\F_q(y_0)$};
			\node[blue] (K1) [above right of=Q]{$\F_{q^\ell}=\F_q(x_1)$};
			\node (Q1) [above right of=Q0]{$\K_1=\K_0(y_1)$};
			\node[blue] (K2)[above right of=K1]{$\F_{q^{\ell^2}}=\F_q(x_2)$};
			\node (Q2) [above right of=Q1]{$\K_2=\K_1(y_2)$};
			\node[blue] (Koo)[above right of=K2]{$\quad \F_q^{(\ell)}$};
			\node (Qoo) [above right of=Q2]{$\quad\K_0^{(\ell)}$};
			%--------------------------
			\draw (Q) edge[red] node[auto]{$r$} (Q0)
			          edge[blue] node[auto,swap]{$\ell$} (K1)
			      (K1) edge[red] node[red, auto] {$r$} (Q1)
			           edge[blue] node[auto,swap]{$\ell$} (K2)
			      (Q1) edge node[auto]{$\ell$} (Q0)
			      (Q2) edge[red] node[red, auto, swap]{$r$} (K2)
			           edge node[auto]{$\ell$} (Q1)
			           edge[dotted] (Qoo)
			      (Koo) edge[blue, dotted] (K2)
			           edge[red] node[red, auto]{$r$} (Qoo);
			\node[below right = 1mm of K2, text width = 4cm, fill = red!10, rounded corners, font = \small]{
				\begin{itemize}
					\item $\K_i= \K_0[Y_i]/\langle Y_i^{\ell^i}-y_0\rangle$
					\item $x_i = \Tr_{\K_i / \F_{q^{\ell^i}}}(y_i)$
					\item We are interested in $Q_i = \operatorname{minpoly}_{\F_q}(x_i)$
				\end{itemize}
			};
		\end{tikzpicture}
	\end{center}
\end{frame}

%###################################################

\begin{frame}
	\frametitle{$\mathbb{G}_m$ and Chebyshev-type extensions}
	
	\begin{exampleblock}{$\mathbb{G}_m$-type extensions}
		\begin{itemize}
			\item $\ell \mid q-1$, $r = 1$.
			\item Initialization: $O_e(\log(q))$ operations in $\F_q$.
			\item $Q_i=X_i^{\ell^i}-y_0 \quad\text{and}\quad T_i=X_{i}^\ell-X_{i-1}$.
			\item Lift and push: no operation in $\F_q$.
		\end{itemize}
	\end{exampleblock}
	
	\vspace*{3mm}
	\begin{exampleblock}{Chebyshev-type extensions}
		\begin{itemize}
			\item $\ell \mid q + 1$, $r = 2$.
			\item Initialization: $O_e(\log(q))$ operations in $\F_q$.
			\item $Q_i(X_i) = Y^{\ell^i} + Y^{-\ell^i} - x_0 \mod Y^2-X_iY+1$.
			\begin{itemize}
				\item $Q_i$ can be computed in $O(\Mult(\ell^i))$; a better algorithm \textcolor{blue}{later}.
			\end{itemize}
			\item Lift and push: \textcolor{blue}{later}.
		\end{itemize}
	\end{exampleblock}
	
\end{frame}

%###################################################

\begin{frame}
	\frametitle{General extensions}
	
	\begin{itemize}
		\item No assumption on $r$ and $\ell$.
		\item \textcolor{red}{\it idea}: change of basis via $\Psi_i$
	\end{itemize}
	
	\begin{center}
		\begin{tikzpicture}
			\footnotesize
			\node (A) {$\F_{q^{\ell^{i}}}$};
			\node[below left=1cm of A] (B) {$\K_i=\K_0[Y_i]/\langle Y_i^{\ell^i}-y_0\rangle$};
			\node[below right=1cm of A] (C) {$\K_0[X_i]/\langle Q_i\rangle$};
			\node[below=1.5cm of B] (D) {$\{y_i^e \mid 0 \le e < \ell^i\}$};
			\node[below=1.5cm of C] (E) {$\{x_i^e \mid 0 \le e < \ell^i\}$};
			%
			\path[->] (A) edge node[above,sloped,inner sep=0.5pt] {$ \sim $} (B);
			\path[->] (A) edge node[above,sloped,inner sep=0.5pt] {$ \sim $} (C);
			\path[->] (B) edge node[above,inner sep=0.5pt] {$ \sim $} node[below,inner sep=2pt] {$\Psi_i$} (C);
			\path[->] (B) edge node[fill = white] {has basis} (D);
			\path[->] (C) edge node[fill = white] {has basis} (E);
		\end{tikzpicture}
	\end{center}
	
	\begin{itemize}
		\item \textcolor{red}{\it result}: 
		\begin{itemize}
			\item $Q_i$ can be computed in $O(i^2\Mult(\ell^{i+1})\log(\ell))$.
			\item $\Psi_i$ or its inverse can be computed in $O(i\Mult(\ell^{i+1}))$. 
		\end{itemize}
		operations in $\K_0$
	\end{itemize}
\end{frame}

%**********************************************************
%**********************************************************

\section{Towers from irreducible fibers}


\begin{frame}
	\frametitle{\insertsection}
	
	\begin{block}{}
		The original idea is due to Couveignes and Lercier, 2011.
	\end{block}
	
	\vspace*{3mm}
	\begin{exampleblock}{Bottom line}
		\begin{itemize}
			\item Let $\phi:G' \rightarrow G$ be a surjective, separable morphism of algebraic 
			$\F_q$-groups.
			\item There are points $x \in G$ with fiber $G'_x$ of cardinality $\ell$.
			\item Assume $G'(\F_q) \rightarrow G(\F_q)$ is not surjective:
			 \begin{itemize}
				 \item $\exists x \in G(\F_q)$ with fiber $G'_x$ in a finite extensions of $\F_q$.
			 \end{itemize}
			\item For appropriate $\phi$, $G'_x$ is irreducible.
			\item Apply a linear projection to $G'_x$.
			\item Repeat with a new map $\phi':G''\to G'$.
		\end{itemize}
	\end{exampleblock}
\end{frame}

%###################################################

\begin{frame}
	\frametitle{Towers from algebraic tori}
	
	\begin{definition}
		For $k=\F_q$, $L=\F_{q^n}$ define 
		\begin{equation*}
			\label{eq:Tn}
			\mathbb{T}_n(k) \cong \{\alpha\in L^\ast \;|\; \Norm_{L/F}(\alpha) = 1 \text{ for all } k\subset F\subsetneq L \}.
		\end{equation*}
		Then $\mathbb{T}_n \cong \mathbb{G}_m^{\euler(n)}$ over $\overline{\F_q}$.
	\end{definition}
	\vspace*{5mm}
	\begin{exampleblock}{$\mathbb{G}_m$-type extensions: $\mathbb{T}_1 \cong \mathbb{G}_m(\F_q)$}
		Suppose $\ell \mid q-1$, $\eta \in \F_q$ a non $\ell$-adic residue, and
		\[
		\setlength{\arraycolsep}{2pt}
		\begin{array}{cccc}
			\phi: & \mathbb{T}_1 & \longrightarrow & \mathbb{T}_1 \\
			& X & \longmapsto & X^\ell
		\end{array}
		\]
		For any $i>0$, the fiber $\phi^{-i}(\eta)$ is $X^{\ell^i}-\eta$.
	\end{exampleblock}
\end{frame}

%###################################################

\begin{frame}
	\frametitle{Towers from algebraic tori (Pell conics)}

	\tikzset{
		point style/.style={circle, inner sep=0.9pt, fill = blue},
		extended line/.style={shorten >= -5mm, shorten <= -5mm}
	}

	\begin{center}
		\begin{tikzpicture}[scale = 0.9, every node/.style={scale=0.9}]
			\draw[name path = pellconic] (-1.5, 2.25) parabola bend (0, 0) (2.1, 4.41);
			\draw[style=help lines] (-1.7, 0) grid (2.2,4.5);
			%-----------
			\coordinate (N) at (1.3, 1.69);
			\coordinate (P) at (1.7, 2.89);
			\coordinate (Q) at (-1.3, 1.69);
			\coordinate (PQtemp) at ($(Q) + (N) - (P)$);
			%-----------
			\draw[extended line] (P)--(Q);
			\path[name path = pline] (PQtemp)--(N);
			\path [name intersections={of = pellconic and pline}];
			\coordinate (PQ) at (intersection-1);
			\draw [extended line] (N) -- (PQ);
			%-----------	
			\node[point style, fill = red, label=below right:$N$] at (N) {};
			\node[point style, label=below right:$P$] at (P) {};
			\node[point style, label=below left:$Q$] at (Q) {};
			\node[point style, label=below left:$P + Q$] at (PQ) {};

			\node[right = 4mm of current bounding box, text width = 7cm, fill = red!10, rounded corners, font = \small] (explain) {
				\textcolor{blue}{\it Pell conic:}
				\begin{equation*}
					\label{eq:Pell}
					C \;:\; x^2 - \Delta y^2 = 4
				\end{equation*}
				
				\vspace*{1cm}
				\textcolor{red}{Addition:} For $P=(x_1,y_1)$ and $Q=(x_2,y_2)$,
				\begin{equation*}
					P\oplus Q = \left(\frac{x_1x_2 + \Delta y_1y_2}{2},\; \frac{x_1y_2 + x_2y_1}{2}\right)
				\end{equation*}
			};
		\end{tikzpicture}	
	\end{center}
\end{frame}

%###################################################

\begin{frame}
	\frametitle{Towers from algebraic tori}
	
	\begin{lemma}
		\begin{itemize}
			\item $\mathbb{T}_2 \cong C$.
			\item Let $P=(\alpha,\beta) \in C$. The abscissa of $[n]P$ is given by $C_n(\alpha)$, where $C_n\in\Z[X]$ is the $n$-th \textcolor{blue}{Chebyshev polynomial}.
		\end{itemize}
	\end{lemma}
	
	\begin{theorem}
		\label{th:T2-irred}
		Let $\eta\in\F_q(\delta)$ be a non $\ell$-adic residue in $\mathbb{T}_2$,
		let $P=(\alpha,\beta)$ be its image in $C/\F_q$. Define
		\[
		\setlength{\arraycolsep}{2pt}
		\begin{array}{cccc}
			\phi: & C/\F_q & \longrightarrow & C/\F_q \\
			& Q & \longmapsto & [\ell^i]Q
		\end{array}
		\]
		Then the abscissas of the fiber $\phi^{-1}(P)$ are the roots of the \textcolor{blue}{irreducible} polynomial $C_{\ell^i}-\alpha$.
	\end{theorem}

	\begin{corollary}
		\begin{equation*}
			Q_i(X_i) = C_{\ell^i}(X_i) - x_0.
		\end{equation*}
	\end{corollary}
\end{frame}

%###################################################

\begin{frame}[fragile]
	\frametitle{Towers from elliptic curves}
	
	\newcommand{\EC}[1]{{\color{blue} \ensuremath{E_{#1}}}}
	\newcommand{\IS}[1]{{\color{red} \ensuremath{\phi_{#1}}}}
	
	\begin{equation*}
		E_0 \;:\; y^2 = x^3 + ax + b, \quad a, b \in\F_q,  \quad \ell \ne p, \quad \ell \mid \#E_0(\F_q)
	\end{equation*}
	
	\begin{block}{}
		Denote by
		\begin{itemize}
			\item $H_0 \lhd E_0/\F_q$ with $\lvert H_0 \rvert = \ell$
			\item $\phi_0$ the unique isogeny with $\ker \phi = H_0$
			\item $E_1$ the image curve of $\phi_0$
		\end{itemize}
	\end{block}
	\vspace*{3mm}
	\begin{center}
		\begin{tikzpicture}
			\def\n{4}
			\foreach \i in {0,...,\n} {
				\pgfmathparse{360/(\n+1)*(\i-1/2) - 90}
				\let\angle\pgfmathresult
				\draw (\angle:1) node (E\i) {\EC{\i}};
				% \draw (\angle:0.4) node {\scriptsize$H_\i$};
				% \draw (\angle:0.7) node[rotate=\angle] {\scriptsize$\subset$};
			}
			\foreach \i in {0,...,\n} {
				\pgfmathparse{int(mod(\i+1, \n+1))}
				\let\j\pgfmathresult
				\draw (E\i) edge[->,bend right=18] node[auto, swap] {\scriptsize \IS{\i}} (E\j);
			}

			\node[left = 3mm of current bounding box.west] (explain) {
			\begin{minipage}{0.7\textwidth}
				Repeat the above with $H_i \lhd E_i/\F_q$ and by $$\IS{i}:\EC{i} \to \EC{i + 1}.$$
				\textbf{\color{orange} Lemma }{\it
				   $\EC{n} \cong \EC{0}$ for some $n \in O(\sqrt{q}\log (q))$.}
			\end{minipage}
			};
			\begin{pgfonlayer}{background}
				\path[fill=yellow!10, rounded corners = 5pt, draw = blue!20] (current bounding box.south west) rectangle
				(current bounding box.north east);
			\end{pgfonlayer}
		\end{tikzpicture}
	\end{center}
\end{frame}

%###################################################

\begin{frame}
	\frametitle{Towers from elliptic curves}
	
	Under the above assumptions, if $P \in E_i$ with $\ell^e \mid \lvert P \rvert$, and $\psi=\phi_{i-1}\circ\phi_{i-2}\circ\cdots\circ\phi_{j}$, then $\psi^{-1}(P)$ is irreducible of cardinality $\ell^{i-j}$.
	\vspace*{5mm}
	\begin{exampleblock}{V\'elu's formulas}
		\begin{equation*}
		  \begin{array}{crcl}
		    \phi_i: & E_i &\longrightarrow & E_{i+1},\\
		    & (x,y) &\longmapsto & \left(\frac{f_i(x)}{g_i(x)}, y\left(\frac{f_i(x)}{g_i(x)}\right)'\right),
		  \end{array}
		\end{equation*}
	\end{exampleblock}
	
	\begin{block}{The $\ell$-adic tower}
		\begin{align*}
		  T_1 &= f_{-1}(X_1) - \eta g_{-1}(X_1),\\ 
		  T_i &= f_{-i}(X_i) - X_{i-1} g_{-i}(X_i).
		\end{align*}
	\end{block}
\end{frame}

%**********************************************************
%**********************************************************

\section{Lifting and pushing}

\begin{frame}
	\frametitle{\insertsection}
	
	\begin{block}{Common structure}
		The	extension at level $i$ is
		$$\F_q[Y]/ \langle S(Y) \rangle \quad\text{and}\quad \F_q[X,Y]/\langle R(X), T(X,Y) \rangle,$$
		with 
		\begin{itemize}
			\item $R$ of degree $\ell^{i-1}$, $S$ of degree	$\ell^i$
			\item $T(X,Y)$ has the form \textcolor{blue}{$f(Y)-X g(Y)$}, with $\deg(f)=\ell$, $\deg(g) < \ell$ and $\gcd(f,g)=1$.
		\end{itemize}
	\end{block}
	
	\begin{definition}
		Let $P \in \F_q[X,Y]$ and $n \in \N$, with $\deg(P,X)< n$. Define
		$$P[f,g,n] = g^{n-1} P\left (\frac fg, Y\right) \in \F_q[Y].$$
	\end{definition}
\end{frame}

%###################################################

\begin{frame}
	\frametitle{Lifting}
	
	\small
	\begin{center}
		\begin{tikzpicture}[every node/.style={scale=0.7}]
			\node [fill = red!10, draw = red!80]{
				\begin{minipage}{0.95\textwidth}
					\begin{algorithm}[H]
						\caption{Compose}
						\label{alg:compose}
						\begin{algorithmic}[1]
							\REQUIRE $P\in \F_q[X,Y]$, $f,g\in \F_q[Y]$, $n\in\N$
							\IF {$n = 1$} 
								\RETURN $P$
							\ELSE
								\STATE $m \la \lceil n/2\rceil$
								\STATE Let $P_0,P_1$ be such that $P = P_0 + X^mP_1$
								\STATE $Q_0 \la$ Compose($P_0, f, g, m$)
								\STATE $Q_1 \la$ Compose($P_1, f, g, n-m$)
								\STATE $Q \la Q_0g^{n-m} + Q_1f^m$  \label{alg:compose:res}
								\RETURN $Q$
							\ENDIF
						\end{algorithmic}
					\end{algorithm}
				\end{minipage}
			};
		\end{tikzpicture}
	\end{center}
	
	\begin{theorem}
	  \label{th:compose}
	  On input $P,f,g,n$, with $\deg(P,X)<n$ and $\deg(P,Y) < \ell$,
	  Algorithm~\ref{alg:compose} computes $Q=P[f,g,n]$ using $O(\Mult(\ell
	  n)\log(n))$ operations in $\F_q$.
	\end{theorem}
\end{frame}

%###################################################

\begin{frame}
	\frametitle{Pushing}
	
	\small
	\begin{center}
		\begin{tikzpicture}[every node/.style={scale=0.7}]
			\node [fill = red!10, draw = red!80]{
				\begin{minipage}{0.95\textwidth}
					\begin{algorithm}[H]
					  \caption{Decompose}
					  \label{alg:decompose}
					  \begin{algorithmic}[1]
					    \REQUIRE $Q,f,g,h\in \F_q[Y]$, $n \in \N$ 
					    \IF {$n=1$} 
						    \RETURN $Q$
					    \ELSE 
						    \STATE $m \la \lceil n/2 \rceil$ 
						    \STATE $u \la 1/ g^{n-m}\bmod f^m$ \label{alg:decompose:xgcd} 
						    \STATE $Q_0 \la Q u \bmod f^m$ 
						    \STATE $Q_1 \la (Q-Q_0 g^{n-m}) {\rm~div~} f^m$ 
						    \STATE $P_0 \la$ Decompose($Q_0, f, g, h, m$) 
						    \STATE $P_1 \la$ Decompose($Q_1, f, g, h, n-m$) 
						    \RETURN $P_0 + X^m P_1$ 
					    \ENDIF
					  \end{algorithmic}
					\end{algorithm}
				\end{minipage}
			};
		\end{tikzpicture}
	\end{center}
	
	\begin{theorem}
	  On input $Q,f,g,h,n$, with $\deg(Q) < \ell n$ and $h = 1/g \bmod f$,
	  Algorithm~\ref{alg:decompose} computes a polynomial $P\in \F_q[X,Y]$
	  such that $\deg(P,X)<n$, $\deg(P,Y) <\ell$ and $Q=P[f,g,n]$ using
	  $O(\Mult(\ell n)\log(n))$ operations in $\F_q$.
	\end{theorem}
\end{frame}

%**********************************************************
%**********************************************************

\section{Implementation}

\begin{frame}
	\frametitle{\insertsection}
	
	\begin{center}
		\includegraphics[width=9cm]{../creat}\\\vspace*{1mm}
		\parbox{7cm}{\scriptsize Times for building $3$-adic towers on top of $\F_2$ (left)
				and $\F_5$ (right), in Magma (first three lines) and using our code.}
	\end{center}
	
	\begin{itemize}
		\small
		\item  We ran tests on an Intel Xeon E5620 clocked at 2.4 GHz, using Sage 5.5 and Magma 2.18.12
		\item Computations that took more than 4GB RAM were	interrupted
	\end{itemize}
\end{frame}

%**********************************************************
%**********************************************************

%\section{Conclusions}
%
%\begin{frame}
%	\frametitle{\insertsection}
%	
%	\begin{itemize}
%		\item We have introduced techniques for
%		\begin{itemize}
%			\item creating $\ell$-adic towers,
%			\item computing embeddings
%		\end{itemize}
%		based on polynomial arithmetic.
%		\item Quasi-linear time and memory in the extension degree.
%		\item Despite its simplicity, our code consistently outperforms Magma on creation time.
%		\begin{itemize}
%			\item Lift and push operations take	essentially no time in Magma.
%		\end{itemize}
%	\end{itemize}	
%\end{frame}








\end{document}









